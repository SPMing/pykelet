%%%%%%%%%%%%%%%%%%%%%%%%%%%%%%%%%%%%%%%%%
% Beamer Presentation
% LaTeX Template
% Version 1.0 (10/11/12)
%
% This template has been downloaded from:
% http://www.LaTeXTemplates.com
%
% License:
% CC BY-NC-SA 3.0 (http://creativecommons.org/licenses/by-nc-sa/3.0/)
%
%%%%%%%%%%%%%%%%%%%%%%%%%%%%%%%%%%%%%%%%%

%----------------------------------------------------------------------------------------
%	PACKAGES AND THEMES
%----------------------------------------------------------------------------------------

\documentclass{beamer}

\mode<presentation> {

% The Beamer class comes with a number of default slide themes
% which change the colors and layouts of slides. Below this is a list
% of all the themes, uncomment each in turn to see what they look like.

%\usetheme{default}
%\usetheme{AnnArbor}
%\usetheme{Antibes}
%\usetheme{Bergen}
%\usetheme{Berkeley}
%\usetheme{Berlin}
%\usetheme{Boadilla}
%\usetheme{CambridgeUS}
%\usetheme{Copenhagen}
%\usetheme{Darmstadt}
%\usetheme{Dresden}
%\usetheme{Frankfurt}
%\usetheme{Goettingen}
%\usetheme{Hannover}
%\usetheme{Ilmenau}
%\usetheme{JuanLesPins}
%\usetheme{Luebeck}
\usetheme{Madrid}
%\usetheme{Malmoe}
%\usetheme{Marburg}
%\usetheme{Montpellier}
%\usetheme{PaloAlto}
%\usetheme{Pittsburgh}
%\usetheme{Rochester}
%\usetheme{Singapore}
%\usetheme{Szeged}
%\usetheme{Warsaw}

% As well as themes, the Beamer class has a number of color themes
% for any slide theme. Uncomment each of these in turn to see how it
% changes the colors of your current slide theme.

%\usecolortheme{albatross}
%\usecolortheme{beaver}
%\usecolortheme{beetle}
%\usecolortheme{crane}
%\usecolortheme{dolphin}
%\usecolortheme{dove}
%\usecolortheme{fly}
%\usecolortheme{lily}
%\usecolortheme{orchid}
%\usecolortheme{rose}
%\usecolortheme{seagull}
%\usecolortheme{seahorse}
%\usecolortheme{whale}
%\usecolortheme{wolverine}


\setbeamertemplate{navigation symbols}{} % To remove the navigation symbols from the bottom of all slides uncomment this line
}

\usepackage{graphicx} % Allows including images
\usepackage{booktabs} % Allows the use of \toprule, \midrule and \bottomrule in tables
\usepackage{amssymb}
\usepackage{amsmath}

\title[Automatic Metadata Extraction]{Automatic Metadata Extraction with Conditional Random Fields} % The short title appears at the bottom of every slide, the full title is only on the title page

\author{Joseph Boyd} % Your name
\institute[EPFL] % Your institution as it will appear on the bottom of every slide, may be shorthand to save space
{
\'Ecole Polytechnique F\'ed\'erale de Lausanne \\ % Your institution for the title page
\medskip
\textit{joseph.boyd@epfl.ch} % Your email address
}
\date{\today} % Date, can be changed to a custom date

\begin{document}

\begin{frame}
\titlepage % Print the title page as the first slide
\end{frame}

%------------------------------------------------

\section{Project Objectives}
\begin{frame}[noframenumbering]{Outline}
\tableofcontents[currentsection]
\end{frame}

%------------------------------------------------

\begin{frame}
\frametitle{Project Objectives}
\begin{itemize}
\item We use Conditional Random Fields (CRFs) to build a probabilistic model
\item CRFs combine ideas from log linear models (logistic regressions) and hidden Markov models (HMMs)
\item Nulla commodo, erat quis gravida posuere, elit lacus lobortis est, quis porttitor odio mauris at libero
\item Nam cursus est eget velit posuere pellentesque
\item Vestibulum faucibus velit a augue condimentum quis convallis nulla gravida
\end{itemize}
\end{frame}

%------------------------------------------------

\section{Theory}
\begin{frame}[noframenumbering]{Outline}
\tableofcontents[currentsection]
\end{frame}

%------------------------------------------------

\subsection{Logistic Regression}
\begin{frame}
\frametitle{Logistic Regression}
\begin{itemize}
\item A logistic regression is used for classifying a data sample into two (binary) or more (multi) categories, thus,
$$\hat{\text{y}}_{prediction} = \boldsymbol\beta^{T} \cdot \boldsymbol{x}_{sample},$$
where $\hat{y}$ is the prediction (represented as a probability), $\boldsymbol{x} = [x_0, x_1, ..., x_D]^T$ is a data sample, and $\boldsymbol\beta = [\beta_0, \beta_1, ..., \beta_D]^T$ is the vector of parameters we must \emph{learn}
\item We construct a (maximum log likelihood) cost function in terms of this parameter vector,
$$\mathcal{L}(\boldsymbol\beta) = \sum_{n=1}^N y_n\boldsymbol\beta^T\boldsymbol{x}_n - \log[1 + \exp(\boldsymbol\beta^T\boldsymbol{x}_n)]$$
\end{itemize}
\end{frame}

%------------------------------------------------

\begin{frame}
\frametitle{Solving a Logistic Regression}\begin{itemize}
\item Building a regression model is equivalent to solving a convex optimisation problem (i.e. maximising the cost function)
\item We know the form of the model, and we have a set of (training) data
\item We want to choose the model parameters for which the error is minimised (think line of best fit)
\item We use a numerical method to obtain the global minimum of error, for example, the method of gradient descent:
$$\boldsymbol\beta^{k+1} = \boldsymbol\beta^{k} - \alpha\nabla\mathcal{L}(\boldsymbol\beta^{k})$$
\end{itemize}
Take home message: we can automatically build mathematical functions for making predictions
\end{frame}

%------------------------------------------------

\subsection{Hidden Markov Models}
\begin{frame}
\frametitle{Hidden Markov Models (HMMs)}
\begin{itemize}
\item Lorem ipsum dolor sit amet, consectetur adipiscing elit
\item Aliquam blandit faucibus nisi, sit amet dapibus enim tempus eu
\item Nulla commodo, erat quis gravida posuere, elit lacus lobortis est, quis porttitor odio mauris at libero
\item Nam cursus est eget velit posuere pellentesque
\item Vestibulum faucibus velit a augue condimentum quis convallis nulla gravida
\end{itemize}
\end{frame}

%------------------------------------------------

\begin{frame}
\frametitle{Hidden Markov Models (HMMs) - Example}
\begin{itemize}
\item Lorem ipsum dolor sit amet, consectetur adipiscing elit
\item Aliquam blandit faucibus nisi, sit amet dapibus enim tempus eu
\item Nulla commodo, erat quis gravida posuere, elit lacus lobortis est, quis porttitor odio mauris at libero
\item Nam cursus est eget velit posuere pellentesque
\item Vestibulum faucibus velit a augue condimentum quis convallis nulla gravida
\end{itemize}
\end{frame}

%------------------------------------------------

\begin{frame}
\frametitle{Solving Hidden Markov Models}
\begin{itemize}
\item Solved using dynamic programming techniques
\item Aliquam blandit faucibus nisi, sit amet dapibus enim tempus eu
\end{itemize}

Take home message: once we have the model, we can make predictions for a given input \emph{efficiently}.

\end{frame}

%------------------------------------------------

\subsection{Conditional Random Fields}
\begin{frame}
\frametitle{Conditional Random Fields (CRFs)}
\begin{itemize}
\item Lorem ipsum dolor sit amet, consectetur adipiscing elit
\item Aliquam blandit faucibus nisi, sit amet dapibus enim tempus eu
\item Nulla commodo, erat quis gravida posuere, elit lacus lobortis est, quis porttitor odio mauris at libero
\item Nam cursus est eget velit posuere pellentesque
\item Vestibulum faucibus velit a augue condimentum quis convallis nulla gravida
\end{itemize}
\end{frame}

%------------------------------------------------

\section{Grobid}
\begin{frame}[noframenumbering]{Outline}
\tableofcontents[currentsection]
\end{frame}

%------------------------------------------------

\section{Initial Results}
\begin{frame}[noframenumbering]{Outline}
\tableofcontents[currentsection]
\end{frame}

%------------------------------------------------

\section{Future Work}
\begin{frame}[noframenumbering]{Outline}
\tableofcontents[currentsection]
\end{frame}

%------------------------------------------------

\end{document} 